\documentclass{homework}

% Palatino for rm and math | Helvetica for ss | Courier for tt
% \usepackage{mathpazo} % math & rm
% \linespread{1.05}        % Palatino needs more leading (space between lines)
\usepackage[scaled]{helvet} % ss
\usepackage{courier} % tt
\normalfont
% \usepackage[T1]{fontenc}
\usepackage{booktabs}
\usepackage{graphicx}
\graphicspath{ {../images/} }
\usepackage{float}
\usepackage{subcaption}

% \usepackage[nottoc]{tocbibind}

\usepackage{biblatex}
\addbibresource{bib.bib}

\usepackage{hyperref}
\usepackage{url}

\Title{Natural Language Processing - Final Project}
\DueDate{December --, 2020}
\ClassName{Intro to Natural Language Processing}
\ClassNumber{CS557}
\ClassSection{Fall 2020}
\Instructor{Professor Jurkat}
\Author{Daniel Kadyrov}

\begin{document}

\maketitle

\begin{abstract}
  \noindent 
\end{abstract}

\newpage
\thispagestyle{empty}
\tableofcontents

\newpage
\setcounter{page}{1}
\section{Introduction}

\section{In vs. On}

\begin{quote}
  Suppose you wanted to automatically generate a prose description of a scene,
  and already had a word to uniquely describe each entity, such as the book, and
  simply wanted to decide whether to use in or on in relating various items, e.g., the
  book is in the cupboard versus the book is on the shelf. Explore this issue by looking
  at corpus data and writing programs as needed. Consider the following examples:
  
  \begin{enumerate}
    \item in the car versus on the train
    \item in town versus on campus
    \item in the picture versus on the screen
    \item in Macbeth versus on Letterman
  \end{enumerate}

  - B and K Chapter 6 Exercise 10 
\end{quote}

\section{Understanding \textit{Natural Language Processing with TensorFlow}}


\end{document}